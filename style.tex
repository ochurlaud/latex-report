%-----Liens PDF-----%
\hypersetup{
%backref=true, %permet d'ajouter des liens dans...
%pagebackref=true,%...les bibliographies
%hyperindex=true, %ajoute des liens dans les index.
colorlinks=true, %colorise les liens
breaklinks=true, %permet le retour à la ligne dans les liens trop longs
urlcolor= blue, %couleur des hyperliens
linkcolor= black, %couleur des liens internes
%bookmarks=true, %créé des signets pour Acrobat
%bookmarksopen=true, %si les signets Acrobat sont créés,
%les afficher complètement.
pdftitle={\title}, %informations apparaissant dans
pdfauthor={\author}, %dans les informations du document
pdfsubject={\sujet} %sous Acrobat.
}

% Not to restart footnote counting in new chapter
\counterwithout{footnote}{chapter}

%----- Définition de l'écart au dessus et en dessous du titre de partie -----%
\def\partecarthaut{3pc}
\def\partecartbas{3pc}

%----- Page 'Titre' -----%
\makeatletter
\def\mytitlepage{
	
\begin{titlepage}
 	\centering

	{\large \textsc{École Centrale de Lyon} \hfill \textsc{\sujet}}
	

  \vspace{2em}

\rule[0.5ex]{\textwidth}{0.1mm}
	\begin{spacing}{2.0}
   {\Large \textbf{\title}}
	\rule[0.5ex]{\textwidth}{0.1mm}

	\end{spacing}
	
\vfill

	\begin{figure}[hp]
	\centering
	\imagecentre
	\end{figure}
	
	\vfill
    \begin{spacing}{1.3}
	\large \author\\
	Supervised by \supervisor \\
	\@date
	\end{spacing}
		
\begin{figure}[h]
\begin{minipage}[c]{2cm}
	\centering
		\imagebasgauche
\end{minipage}
\hfill
\begin{minipage}[c]{5cm}
	\centering
		\imagebasdroit
	\end{minipage}
	\end{figure}
 \end{titlepage}
}
\makeatother

%----- Page 'Remerciements' -----%
\def\myremerciementspage{

		\cleardoublepage
		\vspace*{\fill}
		\chapter*{\centering Thanks}
		\remerciements

		\vspace*{\fill}

		\newpage
}

%----- Page 'Résumé' -----%
\def\myresumepage{
	\vspace*{\fill}

	\chapter*{\centering Abstract}

	\thispagestyle{plain}
	\setlength{\parskip}{0.3cm}

	\begin{center}
	\begin{minipage}[t]{12cm}
	\resumeen


	\centering
		
	\end{minipage}
	\end{center}
	
	\vspace{2cm}
	

	\chapter*{\centering 	\textit{Résumé}}
	
	
	\begin{center}
	\begin{minipage}[t]{12cm}
		\textit{\resumefr}

		
	\end{minipage}
	\end{center}

			
	\vspace{2cm}
	
	\textbf{\centering \small Keywords : \motscles}
	\vfill
	\newpage
}

%----- Page 'conclusion' -----%
\def\conclusionpage{%
	%-- Création du titre --%
	\cleardoublepage
	\vspace*{\partecarthaut}\titlerule\vspace{1.1ex}
	\begin{center}
		\textbf{\LARGE\ CONCLUSION}
		\addcontentsline{toc}{part}{Conclusion}
	\end{center}
	\vspace{2ex}\titlerule\vspace{\stretch{1}}
	\conclusion
	\vspace{\stretch{1}}
	\pagebreak
	}
	
	
\newcommand{\software}[1]{\textit{#1}}


%==== Modification de style général =====%
\makeatletter  

%---- Ecarts numéros table de figure -----%
\newlength{\ecartnumero}
\setlength{\ecartnumero}{2mm}

\dottedcontents{figure}%
  [\dimexpr 15mm+\ecartnumero]
  {}
  {\dimexpr 15mm+\ecartnumero}
  {3.2mm}


%----- Entête du chapitre -----%
\def\@makechapterhead#1{% without *
	\vspace*{10\p@}%
	{\parindent \z@ \raggedright \normalfont
			\interlinepenalty\@M
			\LARGE \bfseries\thechapter \quad#1\par\nobreak
		\vskip 20\p@
	}
}
\def\@makeschapterhead#1{%  with *
	\vspace*{10\p@}%
	{\parindent \z@ \raggedright \normalfont
			\interlinepenalty\@M
			\LARGE \bfseries #1\par\nobreak
		\vskip 20\p@
	}
}

%----- Définition des références -----%
%%---- Sections -----%
\renewcommand{\p@chapter}{\Roman{part}-\thechapter\expandafter\@gobble}
\renewcommand{\p@section}{\Roman{part}-\thesection\expandafter\@gobble}
\renewcommand{\p@subsection}{\Roman{part}-\thesubsection\expandafter\@gobble}
%%---- Equations -----%
\renewcommand\theequation{\Roman{part}-\thechapter.\arabic{equation}}
%%---- Tables et figures -----%
\renewcommand\thetable{\Roman{part}-\thechapter.\arabic{table}}
\renewcommand\thefigure{\Roman{part}-\thechapter.\arabic{figure}}

%----- Tableau multiligne centré avec C{taille}----%
\newcolumntype{C}[1]{>{\centering\arraybackslash }m{#1}}

%====== Définition de frontmatter ========%
\def\frontmatter{%
	\@mainmatterfalse

	\pagenumbering{roman} % Changement du type de numérotation
}
		    
%===== Définition des Annexes =======%
\def\appendix{%
	
  %%----- Espacement entre paragraphes + indentation -----%%
  \setlength{\parskip}{0.3cm}
  \setlength{\parindent}{10pt}
  %%----- Redéfinition des références -----%%
  %%%---- Sections -----%%%
  \renewcommand{\p@chapter}{Appendix \thechapter\expandafter\@gobble}
  \renewcommand{\p@section}{Appendix \thesection\expandafter\@gobble}
  \renewcommand{\p@subsection}{Appendix \thesubsection\expandafter\@gobble}
  %%%---- Equations -----%%%
  \renewcommand\theequation{\thechapter.\arabic{equation}}
  %%%---- Tables et figures -----%%
  \renewcommand\thetable{\thechapter.\arabic{table}}
  \renewcommand\thefigure{\thechapter.\arabic{figure}}

	\cleardoublepage
	\setcounter{page}{1} % Remise à zéro du décompte des pages
	\renewcommand{\thepage}{A-\arabic{page}} % Pagination du type A- xx
	\setcounter{chapter}{0}  % Remise à zéro du décompte des chapitres
	\renewcommand{\theHchapter}{\Alph{chapter}}
	\renewcommand{\thechapter}{\Alph{chapter}} % Numéros de chapitres sous la forme A, B...	
	
	%-- Création du titre --%
	\vspace*{\stretch{1}}\titlerule\vspace{1.1ex}
	\begin{center}
		\textbf{\LARGE\ APPENDIX}
		\addcontentsline{toc}{part}{Appendix}
	\end{center}
	\vspace{2ex}\titlerule\vspace*{\stretch{1}}
	\cleardoublepage
	
}
		
	\renewcommand\chapter{
                    \global\@topnum\z@
                    \@afterindentfalse
                    \secdef\@chapter\@schapter}
								
	\renewcommand\part{
		    \thispagestyle{plain}
                    \global\@topnum\z@
                    \@afterindentfalse
                    \secdef\@part\@spart}
										
	\@addtoreset{chapter}{part}
\makeatother

\titleclass{\part}{top} % make part like a chapter (nouvelle page, pas de saut de page après)
%----- Redéfinition du titre de type \part -----%
\titleformat{\part}
    [display]
    {\centering\normalfont\bfseries\Large}
    {\vspace{\partecarthaut}\titlerule\vspace{1.5ex}\MakeUppercase{\itshape\partname ~ \thepart}}
    {0pt}
	{\vspace{1pc}\LARGE\MakeUppercase}
    [\vspace{2ex}\titlerule\vspace{\partecartbas}]
  
\titlespacing*{\part}{0pt}{0pt}{15pt}

\pagestyle{fancy}
\fancyhf{}

%\let\cleardoublepage\clearpage
\setcounter{chapter}{0}
\setcounter{secnumdepth}{3}
\setcounter{tocdepth}{2}


\makeatletter
	% entre crochets apres le \... : pour lhead, lh-even ou lh-odd ; pour cfoot : cf-even...
	\lhead{\objet}
	\chead{}
	\rhead{\today}
	\renewcommand\headrulewidth{0.5pt}\vspace{0.5cm}

	\vspace{0.5cm}	
	\lfoot{\author}	
	\cfoot{\thepage} %/\pageref{LastPage}}
	\rfoot{\includegraphics[width=110pt]{images/ECL}}
	%\renewcommand\footrulewidth{0.5pt}

\makeatother

%Redéfinit le style 'plain' avec mon style à moi
\fancypagestyle{plain}{}

